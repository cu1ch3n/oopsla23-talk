\documentclass[aspectratio=43,10pt]{beamer}

\usetheme[block=fill]{metropolis}
\usefonttheme[onlymath]{serif}
\setmonofont{iosevka-light.ttf}

\usepackage{appendixnumberbeamer}

\usepackage{booktabs}
\usepackage[scale=2]{ccicons}

\usepackage{pgfplots}
\usepgfplotslibrary{dateplot}

\usepackage{minted}
\usemintedstyle{xcode}

\usepackage{xspace}
\usepackage{xcolor}
\usepackage{pifont}
\usepackage{url}
\usepackage{mathpartir}
\usepackage{amsmath}
\usepackage{mathtools}

\usepackage{ulem}

\usepackage{tikz}
\usetikzlibrary{tikzmark,calc}

\usepackage[symbol]{footmisc}
\renewcommand{\thefootnote}{\fnsymbol{footnote}}

\usepackage{pdfcomment}
\newcommand{\pdfnote}[1]{\marginnote{\pdfcomment[icon=note]{#1}}}
%\newcommand{\pdfnote}[1]{}



\newcommand\name{$F^{b}_{\le}$\xspace}
\newcommand\elementary{$F^{e}_{\le}$\xspace}
\newcommand\mynote[3]{\textcolor{#2}{#1: #3}}

%\newcommand\mynote[3]{}

\newcommand\bruno[1]{~\mynote{Bruno}{red}{#1}}
\newcommand\shengyi[1]{~\mynote{Shengyi}{blue}{#1}}
\newcommand\chen[1]{~\mynote{Chen}{orange}{#1}}

%\newcommand\bruno[1]{}%{\mynote{Bruno}{red}{#1}}
%\newcommand\jimmy[1]{}%{\mynote{Jimmy}{blue}{#1}}

\newcommand\wh{\widehat}

\newcommand{\hquad}{\hspace{0.5em}} % half quad

\newcommand\jg{\omega}                                 % judgment
\newcommand\toto{\rightrightarrows}
\newcommand\To{\Rightarrow}                            % =>
\newcommand\Lto{\Leftarrow}                            % <=
\newcommand\TTo{\mathrel{\mathrlap{\To}\phantom{~}\To}}  % =>>
\newcommand\sto{\rightsquigarrow}
\newcommand\tto{\rightarrowtail}
\newcommand{\elaborate}{\hookrightarrow}
% \newcommand{\elb}[1]{{\setlength\fboxsep{0.5pt}~\colorbox{black!10}{\strut~${\elaborate #1}$~}}}
\newcommand{\elb}[1]{}

\newcommand{\elab}[1]{{\setlength\fboxsep{0.5pt}~\colorbox{black!10}{\strut~${\elaborate #1}$~}}}

\newcommand\Gm{\Gamma}
\newcommand\Om{\Omega}

\newcommand\nil\cdot

% just to fool TexStudio
% \providecommand\inferrule{}

\makeatletter
\let\originferrule\inferrule
\DeclareDocumentCommand \inferrule { s O {} m m o }{%
  \IfBooleanTF{#1}%
  {%
     \mpr@inferstar@[#2]{#3}{#4}%
  }{%
    \mpr@inferrule[#2]{#3}{#4}%
  }%
  \IfValueT{#5}%
  {%
    % #5%
    \my@name@inferrule{#5}%
  }%
}
\NewDocumentCommand \my@name@inferrule { m }{%
  \def\@currentlabelname{\ensuremath{#1}}%
}
\makeatother


% \makeatletter
% \newcommand{\dummylabel}[2]{\def\@currentlabel{#2}\label{#1}}
% \makeatother

\makeatletter
\newcommand{\setword}[2]{%
  \phantomsection
  #1\def\@currentlabel{\unexpanded{#1}}\label{#2}%
}
\makeatother

\newcommand\Mylabel[1]{%
  \zref@labelbyprops{#1}{MyPlainCntValue}%
  \label{#1}%
}%
%
%% This is used for referencing saved MyPlainCntValue-property-values 
%% of zref-labels:
\newcommand\MyPlainCntValueRef[1]{%
   \zref[MyPlainCntValue]{#1}%
}%
\makeatother

\makeatletter
\newcounter{algRuleCounter}
\DeclareDocumentCommand \algrule { o }
{\stepcounter{algRuleCounter}
  \rrule{\arabic{algRuleCounter}}
}
\NewDocumentCommand \my@name@algrule { m }{%
  \def\@currentlabelname{\ensuremath{#1}}%
}
\makeatother


\makeatletter
\newcounter{nalgRuleCounter}
\DeclareDocumentCommand \nalgrule { m }
{\refstepcounter{nalgRuleCounter}
  \rrule{\arabic{nalgRuleCounter}}\Mylabel{#1}
}
\makeatother
\newcommand\refalgrule[1]{\hyperref[#1]{\MyPlainCntValueRef{#1}}}

\newcommand\rto{\longrightarrow}                % "reduce to" arrow ---`
\newcommand\redto{\longrightarrow^*}            % ---`*
\newcommand\rrule[1]{~\longrightarrow_{\makebox[0pt][l]{$\scriptstyle#1$}\hphantom{00}}}

\newcommand\jExt{\rightharpoonup}

\newcommand{\tRed}[1]{\textcolor{red}{#1}}

\newcommand{\lam}{} %[2]{\lambda {#1}.~{#2}}
\RenewDocumentCommand \lam {O{x} m} {\ensuremath{\lambda {#1}.~{#2}}}  % \lambda x.~e --- x is optional. \lam[x]{e}

\newcommand{\tlam}{} %[2]{\lambda {#1}.~{#2}}
\RenewDocumentCommand \tlam  {O{\ta} O{B} m} {\ensuremath{\Lambda ({#1}\le {#2}).~{#3}}}  % \lambda x.~e --- x is optional. \lam[x]{e}

\newcommand{\tlamf}{} %[2]{\lambda {#1}.~{#2}}
\RenewDocumentCommand \tlamf  {O{\ta} m} {\ensuremath{\Lambda {#1}.~{#2}}}  % \lambda x.~e --- x is optional. \lam[x]{e}

\newcommand{\all}{} %[2]{\forall {#1}.~{#2}}
\RenewDocumentCommand \all {O{\ta} O{B} m} {\forall ({#1}\le {#2}).~{#3}}  % \forall a.~A --- a is optional. \all[a]{A}

\newcommand{\allf}{} %[2]{\forall {#1}.~{#2}}
\RenewDocumentCommand \allf {O{\ta} m} {\forall {#1}.~{#2}}  % \

\newcommand{\appInf}[3]{{#1}\bullet{#2}\TTo{#3}}
\newcommand{\appInfAlg}{} %[4]{{#1}\bullet{#2}\TTo_{#3} {#4}}
\newcommand{\absInfAlg}{} %[4]{{#1}\bullet{#2}\TTo_{#3} {#4}}
\NewDocumentCommand \infAlg {O{e} O{\ca} O{\jg}} {{#1}\To_{#2} {#3}}
\RenewDocumentCommand \appInfAlg {m m O{\ca} O{\jg}} {{#1}\bullet{#2}\TTo_{#3} {#4}}
\NewDocumentCommand \tappInfAlg {m m O{\ca} O{\jg}} {{#1}\circ{#2}\TTo_{#3} {#4}}
\RenewDocumentCommand \absInfAlg {O{A} O{\ca} O{\cb} O{\jg}} {{#1}\triangleright_{#2, #3} {#4}}
\newcommand{\tappInf}[3]{{#1}\circ{#2}\TTo{#3}}
\newcommand{\absInf}[3]{{#1}\triangleright{#2}\to{#3}}
% \newcommand{\appInfAlg}{} %[4]{{#1}\bullet{#2}\TTo_{#3} {#4}}
\RenewDocumentCommand \appInfAlg {O{A\to B} O{e} O{\ca} O{\jg}} {{#1}\bullet{#2}\TTo_{#3} {#4}}
% \NewDocumentCommand \tyInfAlg {m m O{a} O{\jg}} {{#1}\circ{#2}\TTo_{#3} {#4}}


\newcommand\ua{}
\newcommand\ub{}
\newcommand\ea{}
\newcommand\eb{}
\newcommand\ta{}
\newcommand\tb{}
\newcommand\sta{}
\newcommand\stb{}

\newcommand\ca{}
\newcommand\cb{}
\newcommand\cc{}
\newcommand\cd{}


\RenewDocumentCommand \ea {O{}} {\wh\alpha_{#1}}   % \widehat{\alpha} --- subscript is optional \al[1]
\RenewDocumentCommand \eb {O{}} {\wh\beta_{#1}}    % \widehat{\beta}

\RenewDocumentCommand \ua {O{}} {\bar a_{#1}}   % \widehat{\alpha} --- subscript is optional \al[1]
\RenewDocumentCommand \ub {O{}} {\bar b_{#1}}   

\RenewDocumentCommand \ta {O{}} {a_{#1}}   % \widehat{\alpha} --- subscript is optional \al[1]
\RenewDocumentCommand \tb {O{}} {b_{#1}}    % \widehat{\beta}

\RenewDocumentCommand \sta {O{}} {\tilde a_{#1}}   % \widehat{\alpha} --- subscript is optional \al[1]
\RenewDocumentCommand \stb {O{}} {\tilde b_{#1}}   

% \RenewDocumentCommand \ca {O{}} { \mathtt{?}a_{#1}}   % \widehat{\alpha} --- subscript is optional \al[1]
% \RenewDocumentCommand \cb {O{}} { \mathtt{?}b_{#1}}   
% \RenewDocumentCommand \cc {O{}} { \mathtt{?}c_{#1}}   

\RenewDocumentCommand \ca {O{}} { a_{#1}}   % \widehat{\alpha} --- subscript is optional \al[1]
\RenewDocumentCommand \cb {O{}} { b_{#1}}   
\RenewDocumentCommand \cc {O{}} { c_{#1}}   
\RenewDocumentCommand \cd {O{}} { d_{#1}}   

\newcommand{\blue}[1]{\textcolor{blue}{#1}}


\newcommand{\newpageVMiddle}[1]{\null\vfill{#1}\vfill}

% \newcommand{\newRule}[1]{\textcolor{purple}{  #1}}

\newcommand{\newRule}[1]{{\setlength\fboxsep{0.5pt}~\colorbox{black!20}{\strut~${ #1}$~}}}

\definecolor{commentgreen}{rgb}{0,0.6,0}

\newcommand{\tint}{\text{\mintinline{ocaml}{Int}}}
\newcommand{\tbool}{\text{\mintinline{ocaml}{Bool}}}

\newcommand{\java}[1]{\lstinline[language=java]{#1}}
\newcommand{\haskell}[1]{\lstinline[language=haskell]{#1}}

\newenvironment{mathparshrink}
  {\begin{@empty}%
    \def \MathparLineskip {\lineskip=0.4em}%
    % \setstretch{0.7}
    % \setlength{\abovedisplayskip}{5pt}
    % \setlength{\belowdisplayskip}{5pt}
    \begin{mathpar}}
  {\end{mathpar}\end{@empty}}

\newcommand{\mtext}[1]{\text{\mintinline{text}{#1}}}
\newcommand{\scala}[1]{\text{\mintinline{scala}{#1}}}
\newcommand{\ocaml}{\mintinline{ocaml}}

\newcommand{\cmark}{\textcolor{mLightGreen}{\ding{51}}}%
\newcommand{\xmark}{\textcolor{mLightBrown}{\ding{55}}}%
\newcommand{\highlight}{\colorbox{lightgray}}
\newcommand{\nl}{\textCR}


\pdfstringdefDisableCommands{%
  \def\\{}%
  \def\texttt#1{<#1>}%
}

\title{Greedy Implicit Bounded Quantification}
\subtitle{}
\author{
    \underline{Chen Cui}, Shengyi Jiang, Bruno C. d. S. Oliveira
}
\date{October 26, 2023}
\institute{The University of Hong Kong
\pdfnote{Good afternoon everyone. My name is Cui Chen. I am currently a PhD student at the University of Hong Kong. Today I am delighted to share our work: Greedy Implicit Bounded Quantification.}
}
\titlegraphic{\raggedleft\includegraphics[width=0.4\linewidth]{logo.png}}

\begin{document}

\maketitle

\begin{frame}[fragile]{Bounded Quantification}
Mainstream OOP languages (Java, Scala, C\#...) have polymorphic type systems with subtyping and \textbf{bounded quantification}
\begin{minted}[escapeinside=||]{java}
public static |\tikzmark{codeblock}\highlight{<S extends Comparable>}| S min(S a, S b) {
    if (a.compareTo(b) <= 0) {
        return a;
    } else {
        return b;
    }
}
\end{minted}
The type of \texttt{min}: \tikzmark{mathblock}\highlight{$\forall{(\texttt{S} \le \texttt{Comparable})}$} ${.~\texttt{S} \to \texttt{S} \to \texttt{S}}$

However, there is little work on \textbf{type inference} algorithms supporting bounded quantification
\begin{tikzpicture}[remember picture, overlay]
    \node[align=center, text width=7cm] (bound) at ($(current page.center)+(2,0.1)$) {
    \begin{block}{Bounded Quantification}
    Gives subtyping bounds to type variables.
    \end{block}
    };
    \draw[->, line width=0.4mm, color=lightgray] ($(pic cs:codeblock)+(3,-0.2)$) .. controls +(0,-1) and +(0,1) .. ($(bound.north)+(0,-0.3)$);
    \draw[->, line width=0.4mm, color=lightgray] ($(pic cs:mathblock)+(1.1,0.4)$) .. controls +(0,1) and +(0,-1) .. ($(bound.south)+(0,0.1)$);
\end{tikzpicture}
\pdfnote{<read 1> Take this Java code as an example. It defines a generic method min. It takes two arguments of type S and returns a value of the same type S. The <S extends Comparable> here defines a type variable S, and gives a bound Comparable to it. We may write the type of min in a formal way like this. Bounded Quantification is used to give subtyping bounds to type variables. So the quantifier is only bounded to range over the subtypes of Comparable. This is a widely used feature in OOP languages. However, ...}
\end{frame}

\begin{frame}[fragile]{Type Inference}
\textbf{Type inference} enables removing redundant type annotations

\begin{minted}[escapeinside=||]{java}
List<String> numbers = Arrays.asList("1", "2", "3", "4", "5");
List<Integer> even = numbers.stream()
                            .|\highlight{map}|(|\highlight{s}| -> Integer.valueOf(s))
                            .filter(number -> number % 2 == 0)
                            .collect(Collectors.toList());
\end{minted}
The type of \scala{map} in Java is:\\
\mintinline{java}{  <R> Stream<R> map(Function<? super T,? extends R> mapper)}
\begin{itemize}
    \item Type argument inference: \scala{map} is instantiated with type \scala{R = Integer}
    \item Argument inference: \scala{s} has type \scala{String}
\end{itemize}
\pdfnote{Type inference enables removing redundant type annotations from programs and can reduce the burden of programmers. When you write the code, the compiler fills the missing types for you and keeps your code very concise and more readable.\nl
Take the code also in Java as an example, it processes a list of strings representing numbers, and converts the strings into numbers,
and filters the even numbers. \nl
In this code, we don't need to provide the type annotations when using the function map, filter, etc. One thing to note here is that the list functions are polymorphic. Here is the type of the map function in Java.\nl
It introduces a type parameter R, takes a function from T to R, and returns a stream of type R. By type inference, Java will automatically instantiate R with type Integer, since the return type of the valueOf function is Integer.\nl
And it can also infer that s has type String, since numbers is a list of Strings. The algorithm will propagate the known type information and conclude that s has type String.
}
\end{frame}


\begin{frame}[fragile]{Research on OOP Type Inference}
Surprisingly little work devoted to practical OOP type inference
\begin{itemize}
    \item Most production compilers (Java/C\#, etc) use algorithms loosely based on:
    \begin{itemize}
        \item {\includegraphics[scale=.8]{beamericonarticle}} Benjamin C. Pierce, David N. Turner.\\
        \qquad \textbf{Local type inference}. TOPLAS 2000.
    \end{itemize}
    \item Scala 2 is based on an improved form of Local type inference:
    \begin{itemize}
        \item {\includegraphics[scale=.8]{beamericonarticle}} Martin Odersky, Christoph Zenger, Matthias Zenger.\\
        \qquad \textbf{Colored local type inference}. POPL 2001.
    \end{itemize}
\end{itemize}
\textbf{Local type inference} suffers from some limitations. Next we will identify these limitations in Scala 2\footnote{Scala 3 has some improvements, but it has not been formally studied.}, and compare it with \name
\pdfnote{
Type inference is very important in a modern programming language. But surprisingly, there is little work devoted to practical OOP type inference. Currently, most production compilers (Java/C\#, etc) use algorithms loosely based on local type inference. And their algorithms are not formally studied and some problems regarding unsoundness and incompleteness have been found.\nl
Scala 2 is based on an improved form of Local type inference, namely colored local type inference. Though Local type inference is the most widely used technique for languages with bounded quantification. It suffers from some limitations.
	Next, we will identify these limitations in Scala 2, and compare it with our calculus Fsub b.
}
\end{frame}

\begin{frame}[fragile]{No support for interdependent bounds}
Local type inference \alert{forbids} the inference of type arguments with interdependent bounds

Scala 2 provides some basic support but it \alert{fails} frequently

\scala{def idFun[A, B <: A => A](x: B): (A => A) = x}
\scala{def idInt1: (Int => Int) = (x => x)}

\pause

\xmark~ In Scala 2, function \scala{idInt2} \alert{fails} to type check:
\scala{def idInt2 = idFun(idInt1)}
\begin{itemize}
    \item \scala{A} is instantiated to $\bot$; \scala{B} is instantiated to $\tint \to \tint$
    \item \xmark~ \scala{B <: A => A} is not true: $\tint \to \tint \le \bot \to \bot$ is not true
\end{itemize}

\pause

\cmark~ In \name, interdependent bounds are supported:

\textmintinline{ocaml}{let idFun }$: \all[\ta][\top]{}\all[\tb][\ta \to
\ta]{}\tb \to \ta \to \ta =\Lambda\ta.~\Lambda\tb.~\lam x$, \\
\textmintinline{ocaml}{  idInt}$:\tint \to \tint = \lam
        x$\textmintinline{ocaml}{ in idFun idInt}

\pdfnote{The first limitation is No support for interdependent bounds. Local type inference forbids the inference of type arguments with interdependent bounds, because they could not find a complete algorithm that deals with such kind of bounds. Scala 2 does provide some basic support for interdependent bounds, but it fails frequently.\nl
If we define two identity functions, idFun is from type B to type A to A, but the bound of B depends on A, another one is from Int to Int. In Scala 2, we cannot apply idFun to idInt1, because Scala 2 wrongly instantiates A with bottom instead of Int. B is instantiated to (Int -> Int).\nl
It does not conform to the bound restriction that B is the subtype of A to A, so it is rejected.
In our calculus, interdependent bounds are supported. It types the equivalent program below.
}
\end{frame}


\begin{frame}[fragile]{Hard-to-synthesize arguments}

\begin{minted}{scala}
def map[A, B](f: A => B, xs: List[A]): List[B] = ...
\end{minted}

\xmark~ In Scala 2, function \scala{mapPlus1} \alert{fails} to type check\\
\scala{def mapPlus1: List[Int] = map(x => 1 + x, List(1, 2, 3))}

As an argument, \scala{x => 1 + x} must have its type synthesized first,
but we can never synthesize its type in local type inference

\pause

\cmark~ Workaround: Provide \textbf{type annotations} to the function argument\\
\mintinline[escapeinside=||]{scala}{def mapPlus2: List[Int] = map((x|\highlight{: Int}|) => 1 + x, List(1, 2, 3))}

\pause

\cmark~ \name can type the program without annotations

\textmintinline{ocaml}{let map}$:\all[\ta][\top]{\all[\tb][\top](\ta \to
      \tb) \to [\ta]\to [\tb]} = \texttt{...}$ \\
      \textmintinline{ocaml}{	 in map}$~(\lam {x + 1})~ [1,2,3]$

\pdfnote{The second limitation is Hard-to-synthesize arguments. We may define a map function in Scala like this. It has 2 type arguments, and takes a function from type A to type B, and a list of A, and returns a list of B. However, when it is applied to an anonymous function. It fails to type check. That is because, as an argument, the anonymous function must have its type synthesized first, but we can never synthesize its tyoe in local type inference.\nl
A workaround is providing type annotations to the function argument. Like we can annotate x with Int here.\nl
Our calculus can type the program without annotations.
}
\end{frame}

\begin{frame}[fragile]{No best argument}
Sometimes the type variable to instantiate appears \textbf{invariantly} in the output type, but the constraints are not enough to decide a \textbf{unique} instantiation
\begin{minted}{scala}
def snd[A]: (Int => A => A) = (x => y => y)
def id = snd(1)
\end{minted}
\xmark~ In Scala 2, the type of \scala{id} is inferred as $\bot \to \bot$. Thus \scala{id} cannot be applied further

\cmark~ In \name, \ocaml{snd}$~1$ can be applied further by deferring the unification\\
\ocaml{let snd }$: \all[\ta][\top]{}\tint \to (\ta \to \ta) \to \ta \to \ta = \Lambda\ta.~\lam{}\lam[y] y$ \ocaml{ in snd}$~1$\\
\pdfnote{When the type variable to instantiate appears invariantly in the output type, but the constraints are not enough to decide a unique instantiation, local type inference fails to provide any instantiations. In such cases, Scala 2 still infers a type, but the type can be meaningless. \nl
For example, when applying the function snd to 1, the constraints are not enough to decide a unique instantiation of A. Scala 2 still infers the type of id as bottom to bottom. But it is meaningless, id cannot be applied further.\nl
While in our calculus, snd 1 can be applied further by deferring the unification.
}
\end{frame}


\begin{frame}[fragile]{Higher-rank type inference}
Take a polymorphic function as the argument of another function
\begin{minted}{scala}
def k(f: Int => Int) = 1
def g(f: ([A <: Int] => A => A) => Int) = 1
\end{minted}
\xmark~ Scala 3\footnote{Scala 2 does not support higher-rank types}, \alert{fails} to type check \scala{def f = g(k)}
\begin{itemize}
    \item \scala{k} has type $(\tint \to \tint) \to \tint$
    \item \scala{g} accepts argument with type $(\all[\ta][\tint]{\ta \to \ta}) \to \tint$
    \item  \xmark~ $(\tint \to \tint) \to \tint \le (\all[\ta][\tint]{\ta \to \ta}) \to \tint$ is rejected
    \begin{itemize}
        \item due to its lack of \textbf{implicit polymorphism}
    \end{itemize}
\end{itemize}

\cmark \name has better support for higher-rank polymorphism  \textmintinline{ocaml}{let k }$: (\tint \to \tint) \to \tint = \lam[f]
      1$, \\
      \textmintinline{ocaml}{	 g}$:((\all[\ta][\tint]{\ta \to \ta}) \to
        \tint) \to \tint = \lam[f] 1$\textmintinline{ocaml}{ in g k}

\pdfnote{Higher-ranked type inference is an important feature. It is very useful when we would like to take a polymorphic function as the argument of another function. However, the subtyping relation employed in local type inference is restrictive and prohibits type-checking many programs with higher-rank types. \nl
In this program, k has type (Int -> Int) -> Int, and g is a function that accepts arugument with this type. We would expect the type inference algorithm can implicitly instantiate A with Int, so that we can apply g to k. However, this fails to type check in Scala 3 due to its lack of implicit polymorphism.\nl
Our calculus has better support for higher-rank polymorphism. It can implicitly instantiate the type variable A here with Int. and the program succeeds to type check.
}
\end{frame}

%\section{\texorpdfstring{\name}{Fsubb} Calculus}

\begin{frame}[fragile]{\name Calculus}

\name extends {$F^{e}_{\le}$\xspace} calculus\footnote{\includegraphics[scale=.8]{beamericonarticle} Jinxu Zhao, Bruno C. d. S. Oliveira. Elementary Type Inference. ECOOP 2022.} with bounded quantification

\begin{itemize}
    \item a variant of kernel $F_\le$
    \begin{itemize}
        \item \textbf{Global} type inference (long-distance constraints)
        \item \textbf{Implicit instantiation} for monotypes (type argument inference)
        \begin{itemize}
            \item $(\tint \to \tint) \to \tint \le (\all[\ta][\tint]{\ta \to \ta}) \to \tint$
        \end{itemize}
        \item \textbf{Explicit type application} for polytypes (\textbf{impredicative polymorphism})
        \begin{itemize}
            \item $(\Lambda \ta.~ \lam{x}: \all[\ta][\top]{\ta \to \ta})~@(\all[\tb][\top]{\tb \to \tb})$
        \end{itemize}
%        \item No inference of \textbf{Top} and \textbf{Bottom} types
%        \item A \textbf{sound}, \textbf{complete} and \textbf{decidable} algorithm
    \end{itemize}
\end{itemize}
Philosophy: infer easy instantiations; use explicit annotations for hard instantiations
\pdfnote{Next we will start to introduce our calculus. Our calculus extends the earlier work (Elementary type inference) with bounded quantification. It is a variant of kernel Fsub. Different from local type inference, it employs global type inference, so it is capable of solving long-distance constraints.\nl
It follows the similar philosophy as Elementary Type Inference, we only infer easy instantiations, and use explicit annotations for hard instantiations. So our system has the implicit instantiation for monotypes. Like the example we showed before, we can implicitly instantiate A with Int. For hard instantiations like polytypes. We use explicit type application to allow users to provide the instantiations.\nl
%Our calculus comes with a sound, complete and decidable algorithm which is formally studied.
}
\end{frame}

\begin{frame}[fragile]{Syntax}
$$ \begin{array}{l@{\quad}lcl}
    \text{Type variables}\quad     & \ta, \tb
    \\[1mm]
    \text{Types}\quad              & A, B, C                      & ::= & \quad
    1 \mid \ua \mid \newRule{\all A}
    \\ &&&\quad \mid A\to B  \mid \top \mid \bot
    \\
    \text{Expressions}\quad        & e, t                         & ::= & \quad
    x \mid () \mid \lam e \mid e_1~e_2 \mid (e:A)
    \\ &&&\quad \mid e~@A \mid
    \newRule{\tlam e : A }
    \\
    \text{Typing contexts}\quad    & \newRule{\tenv} & ::= & \quad
    \nil  \mid \tenv,x:A \mid \newRule{\tenv,\ua\le A}
    \\
    \text{Subtyping contexts}\quad & \Psi                         & ::= & \quad
    \tenv \mid \newRule{\Psi,a \sle A}
    \vspace{0.5em}
  \end{array}$$

Compared with $F_\le^e$, universal types $\all A$ and type abstractions $\tlam e : A$ now
incorporate a bound $B$ to support bounded quantification

\pdfnote{
This is the syntax of our calculus. Compared with elementary type inference. The change is that the Universal types and type abstractions now incorporate a bound B to support bounded quantification.
}
\end{frame}

\begin{frame}[fragile]{Declarative Subtyping Rules}
\framebox{$\Psi \vdash A \le B $} \hfill A is a subtype of B

\begin{mathparshrink}
    \inferrule*[right=$\mathtt{{\le}Unit}$]
    {~}{\Psi \vdash 1 \le 1 \elb{\lam[(x:1)]
        x}}[\mathtt{{\le}Unit}]\label{infrule:sub_unit}
    \and
    \inferrule*[right=$\mathtt{{\le}\top}$]
    {~}{ \Psi \vdash A \le \top
      \elb{\lam[(x:|A|)]{\text{()}}}}[\mathtt{{\le}\top}]\label{infrule:sub_top}
    \and
    \inferrule*[right=$\mathtt{{\le}\bot}$]
    {~}{ \Psi \vdash \bot \le A \elb{\lam[(x:\forall
          a.~a)]{x~@~|A|}}}[\mathtt{{\le}\bot}]\label{infrule:sub_bot}
    \and
    \inferrule*[right=$\mathtt{{\le}Var}$]
    {\ua\ule B\in\Psi\elb{*}}{\Psi\vdash \ua\le \ua \elb{\lam[(x:|a|)]
        x}}[\mathtt{{\le}Var}]\label{infrule:sub_var}
    \and
    \highlight{\inferrule*[right=$\mathtt{{\le}VarTrans}$]
    {\ua\ule B\in\Psi\elb{Co_1: |a \to B|}\quad \Psi\vdash B\le
      A\elb{Co_2}}{\Psi\vdash \ua\le A \elb{\lam[(x:|a|)]
        Co_2~(Co_1~x)}}[\mathtt{{\le}VarTrans}]\label{infrule:sub_vartrans}}
    \and
    \inferrule*[right=$\mathtt{{\le}{\to}}$]
    {\Psi \vdash B_1 \le A_1 \elb{Co_1} \quad \Psi \vdash A_2 \le B_2
      \elb{Co_2}}
    {\Psi\vdash A_1\to A_2 \le B_1\to B_2
      \elb{\lam[(f:|A_1 \to A_2|)]{\lam[(x:|B_1|)] Co_2~(f~(Co_1~x))}}
    }[\mathtt{{\le}{\to}}]\label{infrule:sub_arr}
    \and
    \inferrule*[right=$\mathtt{{\le}\forall L}$]
    {\Psi\vdash^m\tau\quad\Psi\vdash \tau\le B\elb{Co_2}\quad \Psi\vdash
      [\tau/\ta] A \le C \elb{Co_1}\quad  C \text { is not a $\forall$ type} }
    {\Psi\vdash \all A \le C \elb{\lam[(x:|\all A|)]
        Co_1~)((x~@~|\tau|)~Co_2)}}[\mathtt{{\le}\forall L}]\label{infrule:sub_foralll}
    \and
    \inferrule*[right=$\mathtt{{\le}\forall}$]
    {\Psi\vdash B_1\le B_2\elb{*}\quad\Psi\vdash B_2\le B_1\elb{Co_2}\quad\Psi,
      \ta\sle B_2 \vdash A_1 \le A_2 \elb{Co_1}}
    {\Psi\vdash \all[\ta][B_1] A_1 \le
      \all[\ta][B_2]{A_2}\elb{\lam[(x:|\all[a][B_1] A_1|)]{\tlamf{\lam[(f:a\to
              |B_2|)]Co_1~((x~@~a)~(\lam[(x:a)] Co_2~(f~x)
            ))}}}}[\mathtt{{\le}\forall}]\label{infrule:sub_forall}
  \end{mathparshrink}
\pdfnote{
These are the declarative subtyping rules of our system. It is based on standard higher-ranked polymorphic systems. The difference between our system and the elementary type inference is that we have the variable transitivity rule, and all the foralls now incorporate a bound. These two differences cause problems in certain important properties, and motivate us to rethink the question: what kind of type variables should be considered as monotypes?
}
\end{frame}

\begin{frame}{Non-syntactical Monotype}

With bounded quantification, if we treat all type variables as monotypes, transitivity breaks due to rule ${\le}\mathtt{VarTrans}$

\begin{itemize}
  \item[\cmark] $\Psi \vdash A \le B$ :
        \inferrule*[right=\footnotesize{by
            ${\le}\forall\mathtt{L}$}]
        {b \le \forall (c \le 1).~c \to 1 \vdash b \le \top \quad b \le \forall
          (c \le 1).~c \to 1 \vdash b \le b}
        {b \le \forall (c \le 1).~c \to 1 \vdash \forall (a \le \top).~a \le b}
  \item[\cmark] $\Psi \vdash B \le C$ :\\
        $b \le \forall (c <: 1).~c \to 1 \vdash b
          \le \forall (c <: 1).~c \to 1$ {\footnotesize\normalfont\scshape by
            ${\le}\mathtt{VarTrans}$}
  \item[\xmark] $\Psi \vdash A \not\le C$ : bounds are not equivalent\\
        $\forall (a \le \top).~a \not\le
          \forall (c <: 1).~c \to 1$
\end{itemize}

\pause

In \name, only type variables with bound $\top$ or monotype bounds are regarded as monotypes:
\begin{mathparshrink}
\inferrule*[right=$\mathtt{MTVar}$]
    {\ta\le \top \in \Psi}{\Psi \vdash^m \ta}
\and
  \inferrule*[right=$\mathtt{MTVarRec}$]
  {\ta\le A \in \Psi\quad \Psi \vdash^m A}{\Psi \vdash^m \ta}
\end{mathparshrink}

\pdfnote{If we treat all type variables as monotypes, transitivity breaks. type variables with foralls in their bounds, like the type variable b here, can bridge two forall types with different bounds, which is not accepted in our subtyping. So in Fsub b, only type variables with bound top or monotype bounds are regarded as monotypes.}
\end{frame}

\begin{frame}{Two Variants of \name: Complete Algorithm or Not?}

In existing predicative HRP approaches, finding implicit instantiations is \textbf{greedy}. They rely on the property:
$$\tau_1 \le \tau_2 \Longrightarrow \tau_1 = \tau_2$$

\textbf{Variant 1}: \emph{sound}, \emph{complete} and \emph{decidable}
\begin{itemize}
	\item[] Only type variables with \emph{bound $\top$} are regarded as monotypes
\end{itemize}

\textbf{Variant 2}: \emph{sound} but \emph{incomplete}; types more programs
\begin{itemize}
	\item[] Only type variables with \emph{bound $\top$} or \emph{monotype bounds} are regarded as monotypes
	\item[] It breaks the property due to rule ${\le}\mathtt{VarTrans}$:
	\begin{itemize}
		\item[] $a \le \tint \vdash a \le \tint$ but $a \ne \tint$
	\end{itemize}
\end{itemize}


\pdfnote{In existing predicative higher-ranked polymorphism approaches, finding implicit instantiations is greedy.\nl
All such approaches rely on the property that monotype subtyping implies the equality of monotypes. This ensures that if the first instantiation works, then it must also work for subsequent instantiations. However, without care, the property easily breaks in the presence of more expressive subtyping relations, like bounded quantification.\nl
So, in our paper, we present two variants of Fsub b.\nl
In the first variant, only type variables with bound top are regarded as monotypes. With this monotype definition, the property holds, and we get a sound, complete and decidable algorithm.
In the second variant, type variables with monotype bounds are also regarded as monotypes. It breaks the property. For example, in this context, type variable a is a subtype of Int, but a is not Int. so this variant is sound but incomplete. The benefit is that it types more programs.
}
\end{frame}

%\begin{frame}{The Gap between the Declarative and the Algorithmic Systems}
%$$
%\inferrule*[right=$\mathtt{{\le}\forall L}$]
%    {\Psi\vdash^m\newRule{\tau}\quad\Psi\vdash \newRule{\tau}\le B\elb{Co_2}\quad \Psi\vdash [\newRule{\tau}/\ta] A \le C \elb{Co_1}\quad  C \text { is not a $\forall$ type} }
%    {\Psi\vdash \all A \le C \elb{\lam[(x:|\all A|)] Co_1~)((x~@~|\tau|)~Co_2)}}[\mathtt{{\le}\forall L}]
%$$
%
%The monotype $\newRule{\tau}$ is \textbf{guessed} in the declarative system.
%
%In the algorithm, we
%\begin{itemize}
%    \item replace it by a placeholder \textbf{(existential variable)} to solve
%    \item put all the judgements to solve in a unified list (\textbf{worklist}): track the scope easily
%\end{itemize}
%$$\Gamma \Vdash \forall (\ta \le B) .~ A \le C \longrightarrow \Gamma, \newRule{\ea} \Vdash \newRule{\ea} \le B \Vdash [\newRule{\ea} / \ta]A \le C$$
%\pdfnote{... The worklist is like a stack, each time we peek the top of the stack, follow the instructions of the rules, like popping the top out, or add some new todos into the stack. When the worklist becomes empty, we succeed in typing. Otherwise, if it has no rule to apply, we know that the typing fails.}
%\end{frame}

\setlength{\leftmargini}{1.5pt}


\begin{frame}[fragile]{Contributions}
    \begin{itemize}
		\item \textbf{A declarative bidirectional type system}
			\begin{itemize}
				\item Predicative implicit bounded quantification
				\item Impredicative explicit type applications
				\item Checking subsumption, type safety and completeness w.r.t. kernel $F_{\le}$
			\end{itemize}
		\item \textbf{A \textit{sound}, \textit{complete} and \textit{decidable} algorithm of variant 1	}\begin{itemize}
				\item Worklist formulation\footnote{\includegraphics[scale=.8]{beamericonarticle} Jinxu Zhao, Bruno C. d. S. Oliveira, and Tom Schrijvers. A Mechanical Formalization of Higher-Ranked Polymorphic Type Inference. ICFP 2019.}
			\end{itemize}
		\item \textbf{A \textit{sound} algorithm of variant 2 with monotype subtyping}
        \item \textbf{Mechanical formalization and implementation}
              \begin{itemize}
              	\item All theorems are verified in Abella (LOC: 24,919)
              	\item Haskell implementation
              \end{itemize}
    \end{itemize}
\pdfnote{Finally, let me conclude the contributions of our work. Firstly, we proposed a... And for variant 1, we have a sound, complete and decidable algorithm. Our algorithm employs a worklist formulation inspired by previous work. For the second variant with monotype subtyping, we have a sound algorithm. All the theorems have been mechanically formalized in the Abella theorem prover, with about 25 thousand lines of code. and we also have a simple Haskell implementation.}
\end{frame}

\begin{frame}[standout]
	Q\&A
	
	\vspace{5mm}

	\small{Implementation, proofs, and the extended version of the paper are available at: https://doi.org/10.5281/zenodo.8202095}
	
	\vspace{5mm}
	
	\includegraphics[width=0.25\linewidth]{qr-code.png}
\end{frame}

\end{document}